\documentclass[../main.tex]{subfiles}

\usepackage{graphicx}
\graphicspath{{\subfix{../img/}}}

\begin{document}

\chapter{Installation}

\section{Installation de \LaTeX}

\subsection{Windows}

Pour utiliser \LaTeX sous Windows, le plus simple à faire est d'installer un éditeur de texte. J'utilise personnellement \href{https://www.xm1math.net/texmaker/}{\underline{TexMaker}} qui est très complet et spécialement conçu pour écrire des fichier avec \LaTeX.\\
En plus de cet éditeur, il vous faudra installer \href{https://miktex.org/}{\underline{MikTex}} qui le logiciel qui vous permettra de convertir vous fichiers \LaTeX en fichier PDF.

\subsection{Linux}

\lipsum[2 - 4]

\subsection{En ligne}

Si vous cherchez un bon éditeur en ligne, vous pouvez vous tourner vers \href{https://fr.overleaf.com/}{\underline{Overleaf}}. C'est un excellent éditeur qui vous permet notamment de collaborer à plusieurs et en même temps sur un document. Ce site dispose d'une offre gratuite qui vous permet de faire vos premiers pas avec \LaTeX. L'inconvénient de cette solution est donc que si vous voulez accéder à plus de fonctionnalité il vous faudra payer l'abonnement à ce service.\\
Cette solution est donc recommandée si vous avez besoin de travailler avec plusieurs collaborateurs et que cela fait partie de votre métier par exemple.

\section{Git (facultatif)}

Git est un gestionnaire de version qui est très utilisé dans le monde du développement. Utiliser Git dans vos projets \LaTeX vous permettra de travailler avec plusieurs personnes en synchronisant vos fichier avec votre compte GitHub par exemple.

\subsection{Installation de Git}

\subsection{Installation de GitHub}

\subsection{Création de votre Repository GitHub}

\end{document}